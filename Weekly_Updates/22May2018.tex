\documentclass{article}

\title{Macroeconomic Models: \\ Weekly Update}
\author{Chris Comiskey, Open Data Group}
\date{\today}

\usepackage{natbib}
\bibliographystyle{unsrtnat}

\usepackage{fullpage}
\usepackage{ulem}

\usepackage{amsmath, amsthm, amssymb, amsfonts}
\usepackage{mathtools}
\usepackage{float}
\usepackage{bbm}

\usepackage{listings}


\begin{document}

\maketitle{}

\begin{itemize}
\item ``The CPI item structure has four levels of classification. The eight {\bf major groups} are made up of 70 {\bf expenditure classes} (ECs), which in turn are divided into 211 {\bf item strata}.''

\item BLS manual, Chapter 17, Appendix 5: The 4-character descriptors (e.g. FA01) are the item strata; the 2-character descriptors are the 70 expenditure classes; the 1-letter groupings are the 8 major groups. (Shane) \\
e.g. FA031 \\
*F = Food MAJOR GROUP \\
*FA = Cereal EXPENDITURE CLASS \\
*FA03 = Rice/pasta STRATA \\
*FA031 = Rice/pasta ENTRY LEVEL ITEM

\item ``To estimate a relative importance for a component for a month other than December, one can use its previous published relative importance and update it by published price changes.'' \\ (https://www.bls.gov/cpi/tables/relative-importance/home.htm)

\item I calculated the 2010 - 2011 weights by hand---with the December 2009 published weights---and calculated the CPI with those weights and all 68 strata subindices.
    \begin{figure}[H]
    \centering
    \includegraphics[scale=.125]{/Users/cwcomiskey/Desktop/ODG/Macro-models/plots/Weights.jpg}
    \end{figure}
\item Viewing the calculation as a prediction, $\text{R}^{2} = 0.9989$; and a summary of the difference of month-to-month changes between the published CPI and calculated CPI (with recalculated monthy RIWs): 
\begin{verbatim}
   Min. 1st Qu.  Median    Mean 3rd Qu.    Max. 
-0.0452 -0.0233 -0.0116 -0.0108  0.0048  0.0143
\end{verbatim}
\item I think we can attribute the differences to rounding.
\item The next plot adds a RIW25 line to the above plot. The RIW25 line uses only the 25 strata most correlated with CPI.
    \begin{figure}[H]
    \centering
    \includegraphics[scale=.125]{/Users/cwcomiskey/Desktop/ODG/Macro-models/plots/CPI_70_25.jpg}
    \end{figure}
It seems clear that the top 25 strata are not sufficient for calculating/predicting CPI, even with perfect (true) subindices. This motivates further emphasis on collecting data for all strata.
\end{itemize}

\section*{Strata Variability}
    \begin{figure}[H]
    \centering
    \includegraphics[scale=.125]{/Users/cwcomiskey/Desktop/ODG/Macro-models/plots/strata_var.jpg}
    \includegraphics[scale=.125]{/Users/cwcomiskey/Desktop/ODG/Macro-models/plots/24strata_var.jpg}
    \end{figure}



\end{document}