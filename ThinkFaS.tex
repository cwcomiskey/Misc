\documentclass{beamer}

\mode<presentation>
{
  \usetheme{Boadilla}
  \usecolortheme{whale}
  \setbeamercovered{transparent}
  \setbeamertemplate{footline}[frame number]{}
  \setbeamertemplate{navigation symbols}{}
}

\usepackage[english]{babel}
\usepackage[utf8]{inputenc}
\usepackage{times}
\usepackage[T1]{fontenc}
\usepackage{amsmath, amsthm, amssymb, amsfonts}
\usepackage{mathtools}
\usepackage{graphicx}
\usepackage{natbib}
\bibliographystyle{unsrtnat}
\usepackage{float}
\usepackage{bbm}
\usepackage{mathrsfs}

% ======== Alix notation =======
\def\th{\theta}
\def\eps{\epsilon}
\def\beq{\begin{equation}}
\def\eeq{\end{equation}}
\def\bdm{\begin{displaymath}}
\def\edm{\end{displaymath}}
\def\vsd{\vspace{0.1in}}
% ==============================

\usepackage{listings}
\usepackage{relsize}

\makeatletter
\def\verbatim{\scriptsize\@verbatim \frenchspacing\@vobeyspaces \@xverbatim}
\makeatother


\institute{} 
\title{``Thinking, Fast and Slow''}
\subtitle{Evolutionarily Old and New Modes of ``Thinking''}
\author{Book: Daniel Kahneman \\ Colloquium: Chris Comiskey} 

\date{\today} 

\begin{document}

\begin{frame}
  \titlepage
\end{frame}

\begin{frame}{The Book}
\begin{columns}
\column{0.5\textwidth}

\begin{itemize}
\addtolength{\itemsep}{0.5\baselineskip}
\item Understand people a bit better.
\item Understand yourself a bit better.
\item I read it twice.
\end{itemize}

\column{0.5\textwidth}

    \begin{figure}[H]
  	\centering
  	\includegraphics[scale=.25]{tfas.png}
  	\end{figure}
  	
\end{columns}

  	
\end{frame}

\begin{frame}{Introduction}{Kahneman, Evolution, the Brain...}

  \begin{itemize}
  \addtolength{\itemsep}{0.5\baselineskip}
  \item If you don't ``believe'' in evolution...
  \item Kahneman: Behavioral Economics, behavioral psychology, decision making
  \item Introducing: modular theory of mind
    \begin{itemize}
    \addtolength{\itemsep}{0.5\baselineskip}
    \item {\it Kind of} like apps
    \item Apps that interact, compete, interrelate, self-manage(?)
    \item Not a physical partition of brain
    \item e.g. small heat robot
    \item Different parts of the brain evolved at different times, with new functions added on later, melding with existing ones...
    \end{itemize}
  \end{itemize}
\end{frame}
% 
\begin{frame}{Thinking, ``Fast and Slow''}{Old and New}
\begin{itemize}
\addtolength{\itemsep}{0.5\baselineskip}
\item Kahneman calls them ``System 1'' and ``System 2''
\item Evolutionarily old and new
    \begin{itemize}
    \addtolength{\itemsep}{0.5\baselineskip}
    \item Old: amygdala (?) --- fast circuitry; closer to the stock exchange
    \item New: prefrontal cortex --- reflection, abstract thinking, etc.
    \end{itemize}
\item Not the same thing---equating amygdala/prefrontal to fast/slow---but that's the idea behind the more complex underlying brain reality.
\end{itemize}
\end{frame}

\begin{frame}{Fast and Slow: System 1 and System 2}
\begin{itemize}
\addtolength{\itemsep}{0.5\baselineskip}
\item Evolutionarily: old System 1, and new System 2
\item e.g.
    \begin{itemize}
    \addtolength{\itemsep}{0.5\baselineskip}
    \item System 1: ``I can't believe that f****r cut me off!! I'll club him!!''
    \item System 2: ``Then again, maybe he really needed to get over; plus, I don't want to go to jail.''
    \end{itemize}
\item e.g. 2
    \begin{itemize}
    \addtolength{\itemsep}{0.5\baselineskip}
    \item System 1: Recognizing emotions in facial expressions
    \item System 2: 17*34 = ?
    \end{itemize}
\end{itemize}
\end{frame}

\begin{frame}{What's the point?}{}
\begin{itemize}
\addtolength{\itemsep}{0.5\baselineskip}
\item Cognitive biases.
\item Humans commit systematic errors of (rational) thinking.
\item We're irrational in systematic ways.
\item Middle three (of five) sections of book:
    \begin{itemize}
    \addtolength{\itemsep}{0.5\baselineskip}
    \item Heuristics and Biases
    \item Overconfidence (personal favorite)
    \item Choices
    \end{itemize}
\end{itemize}
\end{frame}

\begin{frame}{Heuristics and Biases}{Statistical Thinking}
\begin{itemize}
\addtolength{\itemsep}{0.5\baselineskip}
\item What is a heuristic? A rule of thumb, more or less. (!!)
\item Kahneman and Tversky -- best friends, walks, thought experiments of a kind; so, here we go...
\item ``A study of new diagnoses of kidney cancer in the 3,141 counties of the United States reveals a remarkable pattern. The counties in which the incidence of kidney cancer is lowest are mostly rural, sparsely populated, and located in traditionally Republican states in the Midwest, the South, and the West. What do you make of this?''
\item What {\it do you} make of this?
\end{itemize}
\end{frame}

\begin{frame}{Heuristics and Biases}{Statistical Thinking}
\begin{itemize}
\addtolength{\itemsep}{0.5\baselineskip}
\item ``A study of new diagnoses of kidney cancer in the 3,141 counties of the United States reveals a remarkable pattern. The counties in which the incidence of kidney cancer is lowest are mostly rural, sparsely populated, and located in traditionally Republican states in the Midwest, the South, and the West. What do you make of this?''
\item What {\it do you} make of this?
\item Statistical thinking is hard
\item Were you using System 1 or System 2? Are you sure?
\end{itemize}
\end{frame}

\begin{frame}{Heuristics and Biases}{Statistical Thinking}
\begin{itemize}
\addtolength{\itemsep}{0.5\baselineskip}
\item ``A study of new diagnoses of kidney cancer in the 3,141 counties of the United States reveals a remarkable pattern. The counties in which the incidence of kidney cancer is lowest are mostly rural, sparsely populated, and located in traditionally Republican states in the Midwest, the South, and the West. What do you make of this?''
\item What {\it do you} make of this?
\item Statistical thinking is hard; were you using System 1 or System 2?
\item What about the counties with {\bf highest} incidence of kidney cancer?
\end{itemize}
\end{frame}


\begin{frame}{Heuristics and Biases}{Statistical Thinking}
\begin{itemize}
\addtolength{\itemsep}{0.5\baselineskip}
\item Counties with {\bf highest} incidence of cancer:
  \begin{itemize}
  \item ``...mostly rural, sparsely populated, and located in traditionally Republican states in the Midwest, the South, and the West.''
  \end{itemize}
\item Hmmm...
\end{itemize}
\end{frame}

\begin{frame}{Heuristics and Biases}{Statistical Thinking}
\begin{itemize}
\addtolength{\itemsep}{0.5\baselineskip}
\item Counties with {\bf highest} incidence of cancer:
  \begin{itemize}
  \item ``...mostly rural, sparsely populated, and located in traditionally Republican states in the Midwest, the South, and the West.''
  \end{itemize}
\item Associative, causal-relationship-seeking, story-telling System 1 goes bananas! 
\item Answer: small samples yield extreme results more frequently.
\item What's happening (with System 1) in these situations {\it instead} of statistical thinking? (or whatever other System 2 operations)
\item Let's look at one example System 1 pair in action...
\end{itemize}
\end{frame}

\begin{frame}{Heuristics and Biases}{e.g. Substitution, Availability}
\begin{itemize}
\item Question: which is more likely cause of death, and by what ratio? 
    \begin{itemize}
    \addtolength{\itemsep}{0.5\baselineskip}
    \item Lightning or botulism? 
    \item Accidents or diabetes? 
    \item Disease or accident? 
    \end{itemize}
\item We'll come back to this.
\end{itemize}
\end{frame}

\begin{frame}{Heuristics and Biases}{e.g. Substitution, Availability}
\begin{itemize}
\addtolength{\itemsep}{0.5\baselineskip}
\item Dynamic: brain is lazy, and a System 1 heuristic is easier than calling on System 2.
\item Substitution heuristic: replace a hard question with an easier one
\item e.g. Substitution heuristic, and availability bias
\end{itemize}
\end{frame}

\begin{frame}{Heuristics and Biases}{e.g. Substitution, Availability}
\begin{itemize}
\addtolength{\itemsep}{0.5\baselineskip}
\item Dynamic: brain is lazy, and a System 1 heuristic is easier than calling on System 2.
\item Substitution heuristic: replace a hard question with an easier one
\item e.g. Substitution heuristic, and availability bias
  \begin{itemize}
  \addtolength{\itemsep}{0.5\baselineskip}
  \item How likely is your plane to crash?
  \item How likely is a school shooting at the local school?
  \end{itemize}
\end{itemize}
\end{frame}

\begin{frame}{Heuristics and Biases}{e.g. Substitution, Availability}
\begin{itemize}
\addtolength{\itemsep}{0.5\baselineskip}
\item Substitution heuristic: if a question is too hard, answer an easier one!
\item e.g. substitution heuristic, availability bias
  \begin{itemize}
  \addtolength{\itemsep}{0.5\baselineskip}
  \item How likely is your plane to crash?
  \item How likely is a school shooting at the local school?
  \end{itemize}
\item Common answer much higher than true answer because:
  \begin{itemize}
  \addtolength{\itemsep}{0.5\baselineskip}
  \item Substitution---answer easier question: how easy is it to recall instances of such events?
  \item Availability bias---quite easy, b/c media coverage makes crashes and shootings easy to recall
  \item Ease of recall replaces actual likelihood estimation $\rightarrow$ people think these events are more likely than they are.
  \end{itemize}
\end{itemize}
\end{frame}

\begin{frame}{Heuristics and Biases}{e.g. Substitution, Affect}
\begin{itemize}
\item Which is more likely cause of death, and by what ratio? 
    \begin{itemize}
    \addtolength{\itemsep}{0.5\baselineskip}
    \item Lightning or botulism? lightning:botulism = 
    \item Accidents or diabetes? diabetes:accident = 
    \item Disease or accident? disease:accident = 
    \end{itemize}
\end{itemize}
\end{frame}

\begin{frame}{Heuristics and Biases}{e.g. Substitution, Affect}
\begin{itemize}
\item Which is more likely cause of death, and by what ratio? 
    \begin{itemize}
    \addtolength{\itemsep}{0.5\baselineskip}
    \item Lightning or botulism? lightning:botulism = 52:1
    \item Accidents or diabetes? accident:diabetes = 
    \item Disease or accident? disease:accident = 
    \end{itemize}
\end{itemize}
\end{frame}

\begin{frame}{Heuristics and Biases}{e.g. Substitution, Affect}
\begin{itemize}
\item Which is more likely cause of death, and by what ratio? 
    \begin{itemize}
    \addtolength{\itemsep}{0.5\baselineskip}
    \item Lightning or botulism? lightning:botulism = 52:1
    \item Accidents or diabetes? accident:diabetes = 1:4
    \item Disease or accident? disease:accident = 
    \end{itemize}
\end{itemize}
\end{frame}

\begin{frame}{Heuristics and Biases}{e.g. Substitution, Affect}
\begin{itemize}
\item Which is more likely cause of death, and by what ratio? 
    \begin{itemize}
    \addtolength{\itemsep}{0.5\baselineskip}
    \item Lightning or botulism? lightning:botulism = 52:1
    \item Accidents or diabetes? accident:diabetes = 1:4
    \item Disease or accident? disease:accident = 18:1
    \end{itemize}
\item Example of affect heuristic --- emotional response as probability estimator.
\end{itemize}
\end{frame}


\begin{frame}{Overconfidence}{Causality vs. Chance}
\begin{itemize}
\addtolength{\itemsep}{0.5\baselineskip}
\item People systematically overestimate their understanding of events, and underestimate the role of chance.
\item Kahneman references {\it Fooled by Randomness}, by Nassim Taleb
\item Personal favorite
\item e.g. wealth (pet peeve of mine)
\item $\rightarrow$ ``Geography is destiny.'' -Jack Ryan (Amazon show)
\end{itemize}

\end{frame}

\begin{frame}{Overconfidence}{Hindsight Bias}
\begin{itemize}
\addtolength{\itemsep}{0.5\baselineskip}
\item e.g. Narrative fallacy. 
\item We tell good stories. And we believe them.
\item Why did Google succeed?
  \begin{itemize}
  \item Geniuses, timing, etc.
  \end{itemize}
\item What didn't happen?
\end{itemize}
\end{frame}

\begin{frame}{Overconfidence}{Hindsight Bias}
\begin{itemize}
\addtolength{\itemsep}{0.5\baselineskip}
\item Narrative fallacy. Why did Google succeed?
\item Hindsight bias: we drastically overestimate how well we understand how and why things happened the way they did. 
\item The true test: was it predictable in advance?
\end{itemize}
\end{frame}

\begin{frame}{Overconfidence}{Hindsight Bias}
\begin{itemize}
\addtolength{\itemsep}{0.5\baselineskip}
\item Narrative fallacy. Why did Google succeed?
\item Hindsight bias: we drastically overestimate how well we understand how and why things happened the way they did. 
\item The true test: was it predictable in advance?
\item Google's founders tried to sell for \$1 million, one year in, and failed.
\item Our brains don't deal well with non-events.
\end{itemize}
\end{frame}

\begin{frame}{Overconfidence}{Hindsight Bias}
\begin{itemize}
\addtolength{\itemsep}{0.5\baselineskip}
\item Hindsight bias: we drastically overestimate how well we understand how and why things happened the way they did. 
\item The true test: was it predictable in advance?
\item Our brains don't deal well with non-events.
\item e.g. Imagine: pinpointing three precise pre-fertilization moments---there was a 1/8 chance of a 20th century without Hitler, Stalin, or Mao Zedong. 
\end{itemize}
\end{frame}

\begin{frame}{Overconfidence}{Hindsight Bias}
\begin{itemize}
\addtolength{\itemsep}{0.5\baselineskip}
\item Hindsight bias: we drastically overestimate how well we understand how and why things happened the way they did. 
\item The true test: was it predictable in advance?
\item Our brains don't deal well with non-events.
\item e.g. Imagine: pinpointing three precise pre-fertilization moments---there was a 1/8 chance of a 20th century without Hitler, Stalin, or Mao Zedong. 
\item e.g. 2: In November of 2006 I couldn't decide--for Thanksgiving, should I visit my cousin in Chicago or friend in San Francisco?
  \begin{itemize}
  \item There was a 50/50 chance I would/not meet my wife.
  \end{itemize}
\end{itemize}
\end{frame}

\begin{frame}{Overconfidence}{Hindsight Bias}
\begin{itemize}
\addtolength{\itemsep}{0.5\baselineskip}
\item Hindsight bias: we drastically overestimate how well we understand how and why things happened the way they did. 
\item The true test: was it predictable in advance?
\item Our brains don't deal well with non-events.
\end{itemize}
\end{frame}

\begin{frame}{Choices}{The Prospect of Losses and Gains}
  \begin{itemize}
  \addtolength{\itemsep}{0.5\baselineskip}
  \item Prospect theory (Nobel Price) 
        \begin{itemize}
        \item The {\it prospect} of losses, gains $\rightarrow$ irrationality
        \end{itemize}
  \item Losses loom larger than gains
  \item People are risk averse at the prospect of a loss
  \item e.g. Fill in the blank to make the wager worth it to you:
        \begin{itemize}
        \item 10\% chance losing \$100, 90\% chance winning...
        \end{itemize}
  \end{itemize}
\end{frame}

\begin{frame}{Choices}{The Prospect of Losses and Gains}
  \begin{itemize}
  \addtolength{\itemsep}{0.5\baselineskip}
  \item Prospect theory (Nobel Price) 
        \begin{itemize}
        \item The {\it prospect} of losses, gains $\rightarrow$ irrationality
        \end{itemize}
  \item Losses loom larger than gains
  \item People are risk averse at the prospect of a loss
  \item e.g. Fill in the blank to make the wager worth it to you:
        \begin{itemize}
        \item 10\% chance losing \$100, 90\% chance winning \$12
        \end{itemize}
  \end{itemize}

\end{frame}

\begin{frame}{Choices}{The Prospect of Losses and Gains}
  \begin{itemize}
  \addtolength{\itemsep}{0.5\baselineskip}
  \item Prospect theory (Nobel Price) 
        \begin{itemize}
        \item The {\it prospect} of losses, gains $\rightarrow$ irrationality
        \end{itemize}
  \item Losses loom larger than gains
  \item People are risk seeking at the prospect of a gains
  \item e.g. Fill in the blank to make the wager worth it to you:
        \begin{itemize}
        \item 10\% chance winning \$100, 90\% chance losing...
        \end{itemize}
  \end{itemize}

\end{frame}

\begin{frame}{Choices}{The Prospect of Losses and Gains}
  \begin{itemize}
  \addtolength{\itemsep}{0.5\baselineskip}
  \item Prospect theory (Nobel Price) 
        \begin{itemize}
        \item The {\it prospect} of losses, gains $\rightarrow$ irrationality
        \end{itemize}
  \item Losses loom larger than gains
  \item People are risk seeking at the prospect of a gains
  \item e.g. Fill in the blank to make the wager worth it to you:
        \begin{itemize}
        \item 10\% chance winning \$100, 90\% chance losing \$12
        \end{itemize}
  \end{itemize}

\end{frame}


\begin{frame}{Choices}{The Prospect of Losses and Gains}

  \begin{itemize}
  \addtolength{\itemsep}{0.5\baselineskip}
  \item Prospect theory (Nobel Price) 
        \begin{itemize}
        \item The {\it prospect} of losses, gains $\rightarrow$ irrationality
        \end{itemize}
  \item Losses loom larger than gains
  \item People are risk averse at the prospect of a loss
  \item e.g. Fill in the blank to make the wager worth it to you:
        \begin{itemize}
        \item 10\% chance losing \$100, 90\% chance winning \$12
        \end{itemize}
  \item People are risk seeking at the prospect of a gains
  \item e.g. Fill in the blank to make the wager worth it to you:
        \begin{itemize}
        \item 10\% chance winning \$100, 90\% chance losing \$12
        \end{itemize}
  \end{itemize}

\end{frame}


\begin{frame}{Thinking, Fast and Slow}{}
\begin{columns}
\column{0.5\textwidth}
\begin{itemize}
\addtolength{\itemsep}{0.5\baselineskip}
\item Awesome book
\item Some takeaways: 
  \begin{itemize}
  \addtolength{\itemsep}{0.5\baselineskip}
  \item We're not quite as rational as we think we are
  \item Beware of overconfidence; illusion of understanding, hindsight bias 
  \item We're associative, causality seeing machines
  \item Emotional creatures, emotions influence thinking more than we're consciously aware of; heuristics, substitution 
  \end{itemize}
  \item Thumbtack anecdote
\end{itemize}
\column{0.5\textwidth}
    \begin{figure}[H]
  	\centering
  	\includegraphics[scale=.25]{tfas.png}
  	\end{figure}
\end{columns}
\end{frame}

\begin{frame}{Live Well and Prosper}{}
    \begin{figure}[H]
  	\centering
  	\includegraphics[scale=.2]{Leonard-Nimoy.png}
  	\end{figure}
\end{frame}



\end{document}